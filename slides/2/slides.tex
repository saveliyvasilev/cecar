\documentclass[handout]{beamer}
% Class options include: notes, notesonly, handout, trans,
%                        hidesubsections, shadesubsections,
%                        inrow, blue, red, grey, brown
\usepackage{graphicx}
\usepackage{hyperref}
\usepackage{fancyvrb}
\usepackage{subfig}

\usepackage[utf8]{inputenc}
\usepackage[spanish]{babel}
\usepackage[style=authoryear]{biblatex}
\renewcommand*{\nameyeardelim}{\addcomma\addspace}

\uselanguage{Spanish}
\languagepath{Spanish}

% Theme for beamer presentation.
\usepackage{beamerthemesplit} 
% Other themes include: beamerthemebars, beamerthemelined, 
%                       beamerthemetree, beamerthemetreebars  
% \usetheme{Warsaw}
% \usetheme{Rochester}
\usetheme{Boadilla}

\setbeamertemplate{headline}{%
\leavevmode%
  \hbox{%
    \begin{beamercolorbox}[wd=\paperwidth,ht=2.5ex,dp=1.125ex]{palette quaternary}%
    \insertsectionnavigationhorizontal{\paperwidth}{}{\hskip0pt plus1filll}
    \end{beamercolorbox}%
  }
}

\title[CeCAR]{Utilizando el CeCAR}
% \subtitle{Tesis presentada para optar al título de Doctor de la Universidad de Buenos Aires en el área Ciencias de la Computación}
\author{\texorpdfstring{Saveli Vassiliev\newline\url{vassiliev.sav@gmail.com}}{Saveli Vassiliev}}
% \author[Saveli Vassiliev]{Saveli Vassiliev}\author{\texorpdfstring{Author\newline\url{email@email.com}}{Author}}
\institute[UBA, FCEyN, IC]{Universidad de Buenos Aires \\ Facultad de Ciencias Exactas y Naturales, Instituto de Cálculo}
\date{\today}                    % Enter the date or \today between curly braces

\begin{document}

% Creates title page of slide show using above information
\begin{frame}
  \titlepage
\end{frame}

%\section[Outline]{}
%
%% Creates table of contents slide incorporating
%% all \section and \subsection commands
%\begin{frame}
%  \tableofcontents
%\end{frame}


\begin{frame}
\frametitle{Plan}
\begin{itemize}
  \item<+-> \color{gray} Ayer
  \begin{itemize}
    \item \color{gray} CeCAR (\url{https://cecar.fcen.uba.ar/}) -- ¿de qué se trata?
    \item Creación de claves y usuarios. Usaremos \Verb=PuTTY= para conectarnos al cluster y \Verb=WinSCP= para mover archivos.
    \item Introducción a la terminal de Linux. Aplicaciones de consola que les mejorarán la vida si las aprenden a usar.
  \end{itemize}
  \item<+-> Hoy
  \begin{itemize}
    \item Introducción a comandos de control de \Verb=SLURM=.
    \item Computando $\pi$ usando \Verb=R=: enviando un trabajo, leyendo el output, mirando el estado.
  \end{itemize}
  \item<+-> Próximas dos semanas (para la gente del IC): estoy para resolver sus problemas, documentar los problemas más frecuentes y ayudarles a que todo funcione en sus aplicaciones particulares.
\end{itemize}
\end{frame}

\begin{frame}
\begin{center}
\huge{Comandos de SLURM}
\end{center}
\end{frame}

\begin{frame}[fragile]
\frametitle{sinfo: información general de los nodos del cluster}

\begin{verbatim}
$ sinfo
PARTITION  AVAIL  TIMELIMIT  NODES  STATE NODELIST
...
freeriders    up 3-00:00:00      3  alloc g[03-05]
freeriders    up 3-00:00:00     26   idle e02,g[01-02],n...
batch*        up 3-00:00:00     10    mix e01,g[06-12],n...
batch*        up 3-00:00:00      3  alloc g[03-05]
...
\end{verbatim}
\begin{itemize}
  \item PARTITION: El * denota la partición default.
  \item AVAIL: si está prendido o no (up / down)
  \item TIMELIMIT: tiempo que puede tomar una tarea en esta partición
  \item STATE: down (caído), mix (algunos CPUs ocupados), alloc (ocupado totalmente), idle (disponible).
  \item NODES: nro. de nodos en esa partición en ese estado
  \item NODELIST: los nombres de los nodos
\end{itemize}
\end{frame}



\begin{frame}[fragile]
\frametitle{squeue: muestra los trabajos en el cluster}

\begin{verbatim}
$ squeue
JOBID    PARTITION  NAME   USER  ST       TIME  NODES NODELIST(REASON)
1043049     batch  onion_2 fcarr PD       0:00      4 (Dependency)
1043050     batch  onion_2 fcarr PD       0:00      4 (Dependency)
1043052     batch  onion_1 fcarr  R 1-16:50:11      4 g[09-12]
1042993 batch-gpu rxn_10.4 ldefel R    6:00:04      1 g03
\end{verbatim}
\begin{itemize}
  \item JOBID: número de trabajo/tarea. (Este es importante!)
  \item NAME: nombre del trabajo (es para uds.)
  \item ST: PD (pending), R (running), CA (cancelled), y otros (ver \Verb=man squeue=).
  \item TIME: tiempo que consumió
  \item NODES: nro. de nodos que usa
  \item NODELIST(REASON): los nombres de los nodos que usa, o la razón de por qué aún no está ejecutando (si el estado no es R).
\end{itemize}
\end{frame}



\begin{frame}[fragile]
\frametitle{sbatch: envía un trabajo al cluster}

\begin{verbatim}
$ sbatch slurm.sh
Submitted batch job 1044699
\end{verbatim}
\pause
Y ahora corremos
\begin{verbatim}
$ squeue -u svassiliev
JOBID PARTITION     NAME     USER   ST  TIME  NODES NODELIST(REASON)
1044699_1 batch-xeo simulate svass  R   0:02      1 xeon01
1044699_2 batch-xeo simulate svass  R   0:02      1 xeon01
\end{verbatim}
\end{frame}

\begin{frame}[fragile]
\frametitle{scancel: cancelar un trabajo encolado / corriendo}

\begin{verbatim}
$ scancel 1044699 
$ cat 1044699-1.err
srun: Job step aborted: Waiting up to 32 seconds for job step to finish.
srun: got SIGCONT
slurmstepd: error: *** JOB 1044966 ON xeon06 CANCELLED AT 2018-07-24T14:06:41 ***
srun: forcing job termination
slurmstepd: error: *** STEP 1044966.0 ON xeon06 CANCELLED AT 2018-07-24T14:06:41 ***
srun: error: xeon06: task 0: Terminated
\end{verbatim}
\end{frame}

\begin{frame}[fragile]
\frametitle{scancel: cancelar todos mis trabajos}

\begin{verbatim}
$ scancel -u svassiliev
\end{verbatim}

Cancela todo lo que estoy ejecutando como \Verb=svassiliev=. Para mas opciones mirar \Verb=man scancel=.
\end{frame}

\begin{frame}[fragile]
\frametitle{srun: correr un trabajo}
Sirve para correr trabajos interactivamente en algún nodo. Si alguien necesita esto lo podemos ver otro día.
\end{frame}

\begin{frame}[fragile]
\frametitle{\Verb=--help= y \Verb=--usage=}
Además de tener los manuales \Verb=man <comando>=, podemos ver un sumario de funcionalidad de esta forma:
\begin{verbatim}
$ <comando> --help
$ <comando> --usage
\end{verbatim}

\pause

El \Verb=--help= es muy práctico! Además hay un montón de información en internet si buscan ``SLURM''.
\end{frame}

\begin{frame}
\begin{center}
\huge{Script BASH para enviar trabajos}
\end{center}
\end{frame}


\begin{frame}[fragile]
\frametitle{Estructura del script para enviar trabajos}
\begin{verbatim}
#!/bin/bash

#SBATCH ... (configuración de recursos)
#SBATCH ...

################################

(algunas líneas que piden los administradores,
que no empiezan con #SBATCH)

################################

srun <nuestro programa> <nuestros parámetros>
\end{verbatim}
\end{frame}

\begin{frame}[fragile]
\frametitle{Ejemplo que viene en el CeCAR}
\begin{verbatim}
#!/bin/bash
# Ejecuta un programa común, 
# corriendo 16 tareas distribuidas en 4 nodos.
#SBATCH --job-name="ejemplo programa"
#SBATCH --nodes=4
#SBATCH --ntasks=16
#SBATCH --workdir=/home/USUARIO/ejemplos
#SBATCH --error="ejemplo-%j.err"
#SBATCH --output="ejemplo-%j.out"
#SBATCH --partition=freeriders
#SBATCH --time=10:00

[ -r /etc/profile.d/odin-users.sh ] && . /etc/profile.d/odin-users.sh

srun ./programa
\end{verbatim}
\end{frame}


\begin{frame}[fragile]
\frametitle{Arrays de tareas: Computando $\pi$}
El archivo \Verb=pi/slurm.sh=:
\begin{verbatim}
#!/bin/bash
#
#SBATCH --job-name=simulate_pi
#SBATCH --partition=batch-xeon
#SBATCH --output=%A-%a.out
#SBATCH --error=%A-%a.err
#SBATCH --ntasks=1
#SBATCH --time=00:05
#SBATCH --mem-per-cpu=4096MB
#SBATCH --array=1-100

[ -r /etc/profile.d/odin-users.sh ] && . /etc/profile.d/odin-users.sh

srun Rscript simulate.R $SLURM_ARRAY_TASK_ID
\end{verbatim}
\end{frame}


\begin{frame}[fragile]
\frametitle{Arrays de tareas: Computando $\pi$}
El programa \Verb=pi/simulate.R= empieza con las líneas
\begin{verbatim}
args <- commandArgs(trailing = TRUE)
slurm_array_id <- as.integer(args[1])
\end{verbatim}
Que lee la variable \Verb=$SLURM_ARRAY_TASK_ID= que le pasamos en el script \Verb=slurm.sh= (ver diapo anterior).

Finalmente, desde \Verb=R= escribimos los archivos \Verb=out_<array_id>= usando la variable \Verb=slurm_array_id=.
\begin{verbatim}
write.table(dfrm, sep=",", 
  row.names=FALSE, col.names=FALSE, append=TRUE, 
  file=paste("out", slurm_array_id, sep="_"))
\end{verbatim}
\end{frame}


\begin{frame}[fragile]
\frametitle{Arrays de tareas: Computando $\pi$}
Una vez que termina nuestro proceso (que lo ejecutamos corriendo \Verb=$ sbatch slurm.sh=), veremos muchos archivos generados en la carpeta:

\begin{verbatim}
$ ls
... out_1 out_10 out_100 out_11 ...
... 1045802-51.out 1045802-66.err 1045802-7.out ...
\end{verbatim}
\begin{itemize}
  \item Los \Verb=.err= son archivos de error. Si todo salió bien el comando \Verb=$ cat *.err= no debería mostrar nada.
  \item Los \Verb=.out= son archivos de ``standard output'' (\Verb=stdout=), no deberían tener nada interesante.
  \item Los \Verb=out_<id>= tienen el dato computado por cada \Verb=id= del array:
  \begin{verbatim}
$ cat out_52
52,3.14149016
  \end{verbatim}
  \item Para armar un \Verb=.csv= con todos los $\pi's$ computados: \Verb=$ cat out_* > todos_pi.csv=.
\end{itemize}
\end{frame}



\begin{frame}[fragile]
\frametitle{Moraleja para Monte Carlo en \Verb=R=}
\begin{itemize}
  \item Dividen el número de iteraciones de su programa en \Verb=R= por la cantidad de tareas que decidan correr (menos de 1000 es un buen número).
  \item Su script \Verb=.sh= lo arman editando el \Verb=pi/slurm.sh=. Allí ajustan el tamaño del array.
  \item En su programa \Verb=R= leen el índice del array de SLURM, y lo usan para referenciar lo que deseen. \textbf{Importante:} Escribir en un archivo separado por cada id de array.
  \item Obtienen los resultados de cada nodo leyendo su correspondiente archivo.
  \item Si todo funcionó bien, a disfrutar del speedup!
\end{itemize}
\end{frame}
\end{document}